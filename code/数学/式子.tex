\begin{itemize}
\item 二项式定理及推广
$$
\begin{array}{lr}
\text{对于} \  n \in \mathbb{N} & \left ( 1 + x \right ) ^ {n} = \sum_{i = 0}^{n} \dbinom{n}{i} x^i \\
\end{array}
$$
$$
\begin{array}{lr}
\text{对于} \  \alpha \in \mathbb{R} & \left ( 1 + x \right ) ^ {\alpha} = \sum_{i = 0}^{\infty} \dbinom{\alpha}{i} x^i \\
\end{array}
$$
$$
\text{此处} \dbinom{\alpha}{i} = \frac{\alpha (\alpha - 1) \cdots (\alpha - i + 1)}{i !}
$$
$$
\begin{array}{lr}
\text{对于} \  n \in \mathbb{N}^{*}  & \left ( 1 - x \right ) ^ {-n} = \sum_{i = 0}^{\infty} \dbinom{n + i - 1}{i} x^i \\
\end{array}
$$
\item 二项式反演
$$
g_n = \sum_{i = 0}^{n} \dbinom{n}{i} f_i
\iff
f_n = \sum_{i = 0}^{n} \dbinom{n}{i} (-1)^{n-1} g_i
$$

$$
f(S) = \sum_{S \subseteq T} g(T)
\iff
g(S) = \sum_{S \subseteq T} (-1)^{|T| - |S|} f(T)
$$
\item 范德蒙德卷积
$$
\sum_{i=0}^{k} \dbinom{n}{i} \dbinom{m}{k-i} = \dbinom{n+m}{k}
$$
\item $n$ 个盒子装 $m$ 个球,每个盒子 $ \leq k-1$ 的方案 $F(n, m, k, 0)$

$$
F(n, m, k, x) =
\begin{cases}
0 & \text{ if } x \cdot k > m \text{ or } x > n \\
\binom{n}{x} \cdot \binom{m - x  k + n - 1}{n - 1} - F(n, m, k, x+1) & \text{ else}
\end{cases}
$$

\item 第二类斯特林数
$$
\begin{Bmatrix}
n \\
k
\end{Bmatrix}
= \begin{Bmatrix}
n - 1 \\
k - 1
\end{Bmatrix}
+ k
\begin{Bmatrix}
n - 1 \\
k
\end{Bmatrix}
= \sum_{i = 0}^{k} \frac{ \left ( -1 \right ) ^ {k - i} i ^ n}{ i ! \left ( k - i \right ) ! }
$$
\item 第一类斯特林数
$$
\begin{array}{lr}
\begin{bmatrix} n \\ k \end{bmatrix}
= \begin{bmatrix} n - 1 \\ k - 1 \end{bmatrix}
+ (n - 1) \begin{bmatrix} n - 1 \\ k \end{bmatrix}
& \text{ 边界为 } \begin{bmatrix} n \\ 0 \end{bmatrix} = [n = 0]
\end{array}
$$
\item 卡特兰数
$$
H_n
= \frac{C (2 n, n)}{n + 1}
= \frac{(4 n - 2) H_{n-1}}{n + 1}
= \dbinom{2 n}{n} - \dbinom{2 n}{n - 1}
$$
\item 伯努利数
$$
\begin{array}{lll}
\sum_{j=0}^{m} \binom{m+1}{j} B_j = 0
& (m > 0)
& B_0 = 1
\end{array}
$$
\item 小球装盒问题
\begin{center}
\begin{tabular}{|c|c|c|c|}
\hline
$k$ 个球                & $m$ 个盒子               & 允许有空 & 方案数                                                     \\ \hline
\multirow{4}{*}{各不相同} & \multirow{2}{*}{各不相同} & 是    & $m ^ k \begin{matrix} {\color{white}1} \\ {\color{white}1} \end{matrix} $                                                 \\ \cline{3-4}
                      &                       & 否    & $m !  \begin{Bmatrix} k \\ m \end{Bmatrix}$             \\ \cline{2-4}
                      & \multirow{2}{*}{完全相同} & 是    & $\sum_{i=1}^{m} {\begin{Bmatrix} k \\ i \end{Bmatrix}}$ \\ \cline{3-4}
                      &                       & 否    & $\begin{Bmatrix} k \\ m \end{Bmatrix}$                  \\ \hline
\multirow{4}{*}{完全相同} & \multirow{2}{*}{各不相同} & 是    & $\dbinom{m + k - 1}{m-1}$                               \\ \cline{3-4}
                      &                       & 否    & $\dbinom{k-1}{m-1}$                                     \\ \cline{2-4}
                      & \multirow{2}{*}{完全相同} & 是    & $[x^k] \prod_{i=1}^{m} \frac{1}{1 - x ^ i} \begin{matrix} {\color{white}1} \\ {\color{white}1} \end{matrix}$             \\ \cline{3-4}
                      &                       & 否    & $ [x^{k-m}] \prod_{i=1}^{m} \frac{1}{1 - x ^ i} \begin{matrix} {\color{white}1} \\ {\color{white}1} \end{matrix}$       \\ \hline
\end{tabular}
\end{center}

\item 单位根反演

$$
[n \mid k] = \frac{1}{n} \sum_{i = 0}^{n-1} \omega_{n}^{i k}
$$

$$
\begin{array}{lr}
f(x) = \sum_{k=0}^{m} f_k x^{k}
& \sum_{k=0}^{m} [n \mid k] f_k = \frac{1}{n} \sum_{i=0}^{n-1} f(\omega_{n}^{i})
\end{array}
$$

\item 其他公式

$$
1^k + 2^k + \dots + x^k
= \frac{1}{k+1} \sum_{j=0}^{k} \binom{k+1}{j} B_j \, x^{k+1-j}
$$

$$
d(i \cdot j) = \sum_{p \mid i, p \mid j} \mu (p) \cdot d (\frac{i}{p}) \cdot d (\frac{j}{p})
\qquad
\epsilon (x) = \sum \limits_{d \mid x} \mu (d)
$$

$$
\mathrm{I} * \mathrm{I} = \mathrm{d}
\quad
\mathrm{\varphi} * \mathrm{I} = \mathrm{Id}
\quad
\Lambda * \mathrm{I} = \log
\quad
\mathrm{Id}_k * \mathrm{I} = \mathrm{\sigma}_k
\quad
\mathrm{\mu} * \mathrm{I} = \mathrm{\epsilon}
$$

$$
f * g = h \implies g(1) * S(n) = \sum_{i = 1}^{n} h(i) - \sum_{i = 2}^{n} g(i) S(\left \lfloor \frac{n}{i} \right \rfloor)
$$

$$
f = g * h \implies h(p ^ c) = f(p ^ c) - \sum_{i = 1}^{c} g(p ^ i) \cdot h(p ^ {c-i})
$$
\end{itemize}
